\subsection{What is needed to write a new calculation plugin}
To write a new calc plugin you need the \sh source code.
The calc plugin interfaces and helper classes are stored in the following packages:

\begin{itemize}
  \item \javapackage{edu.udo.scaffoldhunter.plugins}
  \item \javapackage{edu.udo.scaffoldhunter.plugins.datacalculation}
  \item \javapackage{edu.udo.scaffoldhunter.model.data}
  \item \javapackage{edu.udo.scaffoldhunter.model.datacalculation}
\end{itemize}

Your plugin should be placed in a new package called
\javapackage{edu.""udo.""scaffoldhunter.""plugins.""datacalculation.""impl.""PLUGINNAME},
where PLUGINNAME should be replaced by the name of your plugin.

\subsection{Basic parts of a plugin}
Every import plugin consists of three base classes:

  \paragraph{PluginSettingsPanel}
  At first a class on base of \javaclass{PluginSettingsPanel}
  from \javapackage{edu.""udo.""scaffoldhunter.""plugins}.
  This is a \javaclass{JPanel} which shows the configuration options of the plugin
  in the \guidialog{Calc Dialog} (See \figref{fig:createcalcjobdialog}).
  It has methods to get and set the current configuration options
  and those which have been set in the past.

  \paragraph{CalcPluginResults}
  The \javaclass{CalcPluginResults} class is constructed during a plugin run
  and contains all results of the specific plugin with a given configuration.
  With help of the \javaclass{CalcPluginResults} class the calc plugin
  tells the plugin system which properties where calculated and
  supplies an \javainterface{Iterable} over all molecules
  (the calculated properties are attached to those molecules).

  \paragraph{CalcPlugin}
  The calc plugin itself is the interface to \sh.
  It returns instances of the \javaclass{PluginSettingsPanel}
  and the \javainterface{CalcPluginResults},
  has a setter method to get notified about existing properties and
  has getter methods which provide basic information like the title,
  description and unique identifier of the plugin.\\

  Next to those classes there is a \javaclass{PluginArguments} class,
  which is a simple \javaclass{Object} representing a plugin configuration
  and a \javaclass{CalcPluginSettings} class implementing
  the \javainterface{Serializable} interface which contains data,
  that is saved into the database.
  It could be used by the plugin to save and retrieve
  settings like e.g. an arguments history.

\subsection{Writing a simple plugin}
In the next sections we will -- in a step-by-step guide --
develop a plugin that reads an existing numerical property,
adds or subtracts 1.0 from the property
and saves the new value into a new property.
The plugin is configurable and has the ability
to send messages to the GUI if something goes wrong.

\subsection{First version - Create a plugin that does nothing}
The first plugin version can be found in
\javapackage{edu.""udo.""scaffoldhunter.""plugins.""datacalculation.""impl.""example1}.
It implements just the basic things, while having no real functionality.
It can be selected and executed, but behaves transparently,
thus doesn't calculate anything.

  \subsubsection{Example1CalcPlugin.java}

    \paragraph{@PluginImplementation}
    The first important thing is the \javaannotation{@PluginImplementation}
    annotation above the class definition.
    This is needed by the plugin framework to recognize this class as a plugin.
    If this line is missing, the plugin will not be listed in the calc dialog.

    \paragraph{extends AbstractCalcPlugin}
    Your calc plugin has to inherit from \javaclass{AbstractCalcPlugin}.
    This abstract class implements the \javainterface{ImportPlugin} interface
    for you and also overrides the \javamethod{toString()} method so that your plugin title
    is shown correctly in the list of calc plugins in the calc dialog.

    \paragraph{getTitle()}
    The \javamethod{getTitle()} method returns a short name of the plugin.
    This is also the name which is listed in the calc dialog.

    \paragraph{getID()}
    In \javamethod{getID()} your plugin should return a unique name of the plugin,
    this is for example used by \sh to match saved properties for the plugins.
    During the development process of \sh all plugins used
    the form \verb+CLASSNAME_VERSION+ that should be unique enough.

    \paragraph{getDescription()}
    The \javamethod{getDescription()} method returns a description
    about for what the calc plugin can be used,
    like "This is a calc plugin which can be used to calculate the xyz-fingerprint".

    \paragraph{setAvailableProperties(availableProperties)}
    As we do not need to know something about existing properties yet,
    we leave this method blanc.
    This will change in the fourth version.

    \paragraph{getSettingsPanel(settings,arguments)}
    As we do not need any configuration yet,
    we just return an empty instance of \javaclass{SettingsPanel}.
    This will change in the third version.

    \paragraph{getResults(arguments,molecules,msgListener)}
    The given \javavar{molecules} parameter is
    an \javaclass{Iterable} over the available molecules.
    As we do not want to calculate any property nor modify any molecule,
    we return a class implementing \javainterface{CalcPluginResults} interface,
    which will return the \javavar{molecules} parameter unchanged.
    It also returns an empty \javaclass{Set} of \javaclass{PropertyDefintition}s,
    which declares we have no properties to be added to \sh.

\subsection{Second version - 'Calculate' a new property}
The second version of the plugin can be found in
\javapackage{edu.""udo.""scaffoldhunter.""plugins.""datacal""culation.""impl.""example2}.
Here, we change our last plugin version,
so that it creates and saves a property for all given molecules.
We don't really calculate something,
we just create a numerical property with value 1.0.

  \subsubsection{Example2CalcPlugin.java}
    \paragraph{Example2CalcPlugin()}
    In the constructor of \javaclass{Example2CalcPlugin},
    we create a new \javaclass{Property""Definition}
    and store it in a member variable named \javavar{propDef}.
    \javavar{propDef} describes the characteristics of the property we want to add to every molecule.
    Therefore we set the property type to be a numerical property.
    See \tableref{Table:scaffoldhunter:calc:PropertyType} to learn which property types are available.
    We also set a title, a description and a key.
    The title is a short description of the property,
    where as the description is a sentence describing the property in detail.
    The property key is used for internal processing and written in uppercase letters by convention.
    It should be as unique as possible.
    Additionally, we set the property definition to be mappable
    (this means it can be mapped on a visual feature in the main program)
    and define it as molecule property (by saying it is not a scaffold property).

    \begin{table}[!htb]
      \begin{tabular}{cp{10cm}}
	\textbf{PropertyType}	& \textbf{Description} \\ \toprule
	\verb+NumProperty+		& An ordinary numerical property.\\ \midrule
	\verb+StringProperty+	& An ordinary string property.\\ \midrule
	\verb+BitStringFingerprint+	& A bit fingerprint represented by a string of 1 and 0 (chars).\\ \midrule
	\verb+BitFingerprint+	& A bit fingerprint that interprets every bit of a string as a bit. This is logically identical to BitStringFingerprint but has less memory consumption.\\ \midrule
	\verb+NumericalFingerprint+	& A fingerprint that consists of many numerical values. A NumericalFingerprint is a simple string with integer values separated by a comma: int,int,...\\ \bottomrule
      \end{tabular}
      \caption{Property Types}
      \label{Table:scaffoldhunter:calc:PropertyType}
    \end{table}

    \paragraph{getResults(arguments,molecules,msgListener)} \label{sec:scaffoldhunter:calc:Example2CalcPlugin.java:getResults}
    Here we change our custom \javaclass{CalcPluginResults} implementation:
    In the \javamethod{getMolecules()} method we not simply return the \javaclass{Iterable} over the available molecules like in the last version.
    Instead we transform all molecules with a custom transform function first, and return the transformed molecules.
    The transform function will do all the work like calculating a property and adding it to the molecule.
    Read \subsecref{sec:scaffoldhunter:calc:Example2CalcPluginTransformFunction.java} to see what is does in our example.\\
    In the \javamethod{getCalculatedProperties()} method we return a \javaclass{Set} which contains the \javavar{propDef} we created before.
    This notifies the plugin system we want to add this property.

  \subsubsection{Example2CalcPluginTransformFunction.java} \label{sec:scaffoldhunter:calc:Example2CalcPluginTransformFunction.java}

    \paragraph{implements Function$<$Molecule, Molecule$>$}
    Our transform function needs to implement the \javainterface{Function} interface
    and we want to transform from molecule to molecule.

    \paragraph{Example2CalcPluginTransformFunction(propDef)}
    In the constructor of the \javaclass{Example2""Calc""Plugin""Transform""Function} we simply save the given
    \javaclass{PropertyDefinition} parameter in a member variable named \javavar{propDef}.
    We will need this in the \javamethod{apply()} function.

    \paragraph{apply(molecule)}
    The \javamethod{apply()} functions gets one molecule as input parameter, and returns the transformed molecule.
    We just insert a mapping from the \javavar{propDef} (our \javaclass{PropertyDefinition}) to the value $1.0$ to the molecules property map.
    Then we return the molecule. The plugin system will read this map and save the property.

\subsection{Third version - Make the plugin configurable}
The third version of the plugin can be found in
\javapackage{edu.""udo.""scaffoldhunter.""plugins.""datacalculation.""impl.""example3}.
We now want to make the plugin configurable.
The user should choose whether the 'calculated'/added property is set to $1.0$ or $-1.0$, by enabling or disabling a checkbox.
Therefore we will extend \javaclass{PluginSettingsPanel} by a checkbox
and introduce a \javaclass{CalcPluginArguments} class to store the state of the checkbox.
  
  \subsubsection{Example3CalcPluginArguments.java}
  The \javaclass{Example3CalcPluginArguments} class just has a
  \javatype{boolean} member variable encoding the state of the checkbox.
  Additionally it has a getter and a setter method for this variable.

  \subsubsection{Example3CalcPluginSettingsPanel.java}
    \paragraph{Example3CalcPluginSettingsPanel(arguments)}
    The \javaclass{Example3CalcPluginSettingsPanel}s constructor saves
    a reference to the given \javavar{arguments} parameter.
    It also creates a checkbox which is initialized to the state stored
    in the \javavar{arguments} parameter and adds it to the panel.
    Additionally it creates an \javainterface{ActionListener} which reacts on changes
    of the checkbox state and updates the corresponding
    \javatype{boolean} value in the \javavar{arguments}.

    \paragraph{getArguments()}
    The \javamethod{getArguments()} method simply returns the \javavar{arguments}.

  \subsubsection{Example3CalcPluginTransformFunction.java}
  The \javaclass{Example3CalcPluginTransformFunction} constructor is changed
  so that it saves a reference to the new \javavar{arguments} parameter.

    \paragraph{apply(molecule)}
    The \javamethod{apply()} method now determines the property value based on the saved \javavar{arguments}.

  \subsubsection{Example3CalcPlugin.java}
      In the calc plugin itself there are just a few changes. The following methods changed:

      \paragraph{getSettingsPanel(settings,arguments)}
      The \javamethod{getSettingsPanel()} method creates
      a new \javaclass{Example3""CalcPlugin""Arguments} instance,
      if the \javavar{arguments} parameter is \javatype{null}.
      You should always initialize your arguments in this way.
      Afterwards a new \javaclass{Example3""CalcPlugin""SettingsPanel} is
      instantiated with the \javaclass{Example3""CalcPlugin""Arguments} as parameter and returned.

      \paragraph{getResults(arguments,molecules,msgListener)}
      In the \javamethod{getResults()} method there is just one small change:
      The \javavar{arguments} parameter is casted and passed
      to the \javaclass{Example3""CalcPlugin""Transform""Function}s constructor.

\subsection{Fourth version - Use existing properties for calculation}
The fourth version of the plugin can be found in
\javapackage{edu.""udo.""scaffoldhunter.""plugins.""datacalculation.""impl.""example4}.
In this version we want to read existing numerical properties and let the user select one.
For every molecule our plugin creates a new property which is the
same as the selected one, but $1.0$ or $-1.0$ (based on the users choice) is added to the property value.

  \subsubsection{Example4CalcPluginArguments.java}
  A new variable which saves the property chosen by the user is added together
  with corresponding getter and setter methods
  to the \javaclass{PluginArguments} from the last version.

  \subsubsection{Example4CalcPluginSettingsPanel.java}
  The \javaclass{PluginSettingsPanel} from the last version is extended to show a
  \javaclass{JList} with all numerical property definitions.
  A list selection listener is used to update the \javaclass{Example4CalcPluginArguments}
  with the property definition selected by the user.

  \subsubsection{Example4CalcPluginTransformFunction.java}
  The \javamethod{apply()} method was adjusted so that the value of
  the chosen property is read from the input molecule,
  $1.0$ or $-1.0$ is added and the resulting new property value
  is appended to the property map of the output molecule.

  \subsubsection{Example4CalcPlugin.java}
  In comparison to the last version there are several small changes in \javaclass{Example4CalcPlugin}.
  The constructor was deleted and the creation of the
  property definition moved to the \javamethod{getResults()} method.

  \paragraph{setAvailableProperties(availableProperties)}
  In the \javamethod{setAvailableProperties()} method the
  \javavar{available""Properties} are saved as a member variable.
  Please note that the \javamethod{setAvailableProperties()} method is
  the first method called by the plugin system after instantiation of the plugin.
  For this reason the plugin is able to use this information when
  creating a \javaclass{SettingsPanel} in the \javaclass{getSettingsPanel()} method.

  \paragraph{getSettingsPanel(settings, arguments)}
  The only change made in the \javamethod{getSettingsPanel()}
  method is that the \javavar{availableProperties} are passed
  to the constructor of the \javaclass{Example4""CalcPlugin""SettingsPanel}.

  \paragraph{getResults(arguments,molecules,msgListener)}
  The \javaclass{getResults()} method is now responsible for creation
  of the \javaclass{PropertyDefinition} stored in \javavar{propDef}.
  \javavar{propDef}s key, title and description attributes are set dynamically based on the
  corresponding attributes of the chosen input property stored in the parameter \javavar{arguments}.

\subsection{Fifth and last version - Display a message during calculation}
The fifth and last version of the plugin can be found in
\javapackage{edu.""udo.""scaffoldhunter.""plugins.""datacalculation.""impl.""example5}.
Here we will enable the plugin to send messages to the GUI
in case that the input property chosen by the user is not defined for a molecule.

  \subsubsection{Example5CalcPlugin.java}
  In comparison to the last plugin version, there is just one small change:
  The \javamethod{getResults""(arguments,""molecules,""msgListener)} method
  now passes the \javavar{msgListener} parameter to the constructor of
  the \javaclass{Example5CalcPluginTransformFunction}.

  \subsubsection{Example5CalcPluginTransformFunction.java}
  All the message handling is done in the \javaclass{Example5CalcPluginTransformFunction}.
  In its constructor we therefore read the new \javavar{msgListener}
  parameter and save it as a member variable with the same name.
  In the \javamethod{apply()} method we will
  then use the \javavar{msgListener} member variable to send a message.\\
  But first we need more details about the message system integrated into the calc plugin system.
  Because the \javavar{msgListener} variable references an object implementing the \javainterface{MessageListener} interface,
  we are able to send a \javaclass{Message} object by calling \verb+msgListener.receiveMessage(message)+.
  Suited for the needs of calc plugins there is the \javaclass{CalcMessage} class which extends the \javaclass{Message} class.
  \javaclass{CalcMessage} has two similar constructors allowing to create
  a message with a desired \javaclass{MessageType}, a molecule title and (optionally) a property definition.
  An object implementing the \javainterface{MessageType} interface defines a message string and a message icon.
  For convenience there is an enum \javaclass{CalcMessageTypes} with some predefined \javaclass{MessageTypes}, which can be used for your message.
  See \tableref{Table:scaffoldhunter:calc:CalcMessageTypes} to learn what types you can use.
  Depending on the used \javainterface{MessageType} implementation some or all of the given attributes
  of a message are presented to the user in a tree-like manner shown in \figref{fig:calculationprogress}.

    \paragraph{apply(molecule)}
    In the \javamethod{apply()} method an else clause (which sends the message) was inserted
    after the if-block (which does the calculation).
    So in case the chosen property is not defined for the given molecule,
    a \javaclass{CalcMessage} is created and sent to the GUI.
    Note that the molecule title needed to construct the message is read from the molecule properties map,
    by asking for the key \verb+CDKConstants.TITLE+.

  \subsubsection{Example5CalcPluginArguments.java}
  This class remains completely unchanged.

  \subsubsection{Example5CalcPluginSettingsPanel.java}
  This class remains completely unchanged.

    \begin{table}[!htb]
      \begin{tabular}{cp{10cm}}
	\textbf{Enum constant}		& \textbf{Message text} \\ \toprule
	\verb+PROPERTY_NOT_PRESENT+	& $<$molecule title$>$: source property $<$property definition title$>$ needed for calculation was not present, thus no value calculated\\ \midrule
	\verb+CALCULATION_ERROR+	& $<$molecule title$>$: an error occurred in the calculation plugin, thus no value calculated\\ \bottomrule
      \end{tabular}
      \caption{Calc Message Types}
      \label{Table:scaffoldhunter:calc:CalcMessageTypes}
    \end{table}

\subsection{Use transform options to handle frequent tasks}

If you write a calc plugin that takes the structural
information (e.g. 2D graph structure) of a molecule into account,
then you may want to give the user the opportunity to use transform options.
Transform options are a kind of preprocessing on all molecules before the calc plugin operates on them.
See \subsecref{subsec:scaffoldhunter:Transform Options} for more information.\\

Using transform options is rather simple.
First change your \javaclass{PluginArguments} class
so that it inherits from \javaclass{AbstractCalcPluginArguments},
which can be found in \javapackage{edu.""udo.""scaffoldhunter.""plugins.""datacalculation}.
The second thing you have to do, is to integrate an instance of \javaclass{CalcPluginTransformOptionPanel}
(can be found in \javapackage{edu.""udo.""scaffoldhunter.""plugins.""data""calculation})
in your custom \javaclass{PluginSettingsPanel}.
As the name suggests, you can add the \javaclass{CalcPluginTransformOptionPanel} like any other Swing container somewhere to your custom panel.
When instantiating \javaclass{CalcPluginTransformOptionPanel},
you need to pass your \javaclass{PluginArguments} instance to its constructor.
After following the instructions above your plugin will show a panel with transform
options and depending on the users choice the molecules are transformed automatically before they are processed by the plugin.