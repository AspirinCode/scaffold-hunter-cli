Searching chemical space for a particular compound is like looking
for a needle in a haystack. To assist you with this task, the software
\emph{\sh} was developed. \sh uses the concept of \emph{scaffolds}, i.e.,
underlying molecular frameworks that serve as simplified representatives for
classes of similar molecules. These scaffolds can be organized in a tree-like
hierarchy based on the inclusion relation, enabling the user to navigate in
the associated chemical space in an intuitive way.
Furthermore, \sh supports a variety of views allowing the user to inspect 
chemical compound data from different perspectives.

The first step in using \sh is to aggregate data in
a database, and create a so-called \emph{scaffold tree} for this data.
These topics are covered in the first part of the manual. Once this
is done the data can be navigated, filtered, viewed and annotated
by multiple users. This is described in the second part of the manual.

%chapter descriptions here?
\vspace{2.5cm}

\emph{History.} The Scaffold Hunter project was initiated by Stefan Wetzel and Karsten Klein
and a first version of \sh was implemented by the project group 504~\cite{pg504}
at the TU Dortmund in 2009. For version 2.x, many parts of it were improved, extended and rewritten by
the project group 552~\cite{pg552}, resulting in a program with many additional features,
better support of chemical workflows, and improved usability.
Currently, the project is managed by Nils Kriege and Karsten Klein at TU Dortmund, and 
continuously improved in a cooperation of TU Dortmund and the Max-Planck-Institute for Molecular
Physiology.